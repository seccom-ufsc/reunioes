\documentclass{ata-calico}

\begin{document}

\maketitle

\pauta{Escolha de professor responsável}

Foi contextualizado sobre a situação da SECCOM aos novos membros da CO\@: o
projeto na FEESC, que tem o dinheiro da última SECCOM, se encontra no nome da
professora Patrícia Plentz e é necessário passar para o novo professor
responsável. Os professores sugeridos haviam sido: Maicon Rafael Zatelli,
Jerusa Marchi, Delucca, Jean Hauck, Jean Martina e Lúcia Helena. Júlia e
Vanessa apoiaram chamar o Jean Hauck.

Luis Oswaldo comenta que é interessante que o professor responsável deixe a CO
se organizar de maneira independente (evitando interferir nas decisões dela).
Victor comenta que o Jean atuou na computação na escola. Vanessa comenta que
ele também tem um pensamento não apenas acadêmico, tem noção de gestão. Marcelo
comenta que ele é próximo dos alunos. Lucas sugere elencar uma prioridade de
professores com quem conversar.  Sobre a Jerusa, é comentado que ela parece
muito ocupada. Luis comenta que o Maicon passa apenas a impressão de ser
ocupado, mas na verdade é bastante animado com esse tipo de projeto. Vanessa
comenta que o Delucca talvez não aceitasse. Thales comenta que talvez não fosse
tão interessante chamar o Jean Martina, por ele acabar deixando a organização
livre demais (Luis responde que isso não seria um problema). Vanessa sugere
também a professora Cristina Meinhardt. Júlia também sugere o professor Rafael
Santiago, que vai ser da banca do TCC dela, mas Luis avisa que pode não ser
muito bom eleger um professor que não conhecemos tão bem.

Lista de prioridades elencada:

\begin{enumerate}
    \item Jean Hauck;
    \item Maicon Rafael Zatelli;
    \item Cristina Meinhadt.
\end{enumerate}

\textbf{João Paulo ficará responsável de entrar em contato com os professores}.

Luis pergunta se alguém conversou com a professora Plentz sobre a decisão de
passar o nome dela para outro professor e sugere que alguém entre em contato
para formalizar a situação. \textbf{Thales se voluntariou para fazer esse
contato.}

\pauta{Reserva do espaço físico}

Na última reunião, foram sugeridos os espaços: Teixeirão, EPS e Reitoria. Lucas
comenta que para decidir o local é importante ter em mente o que se planeja
fazer na SECCOM\@. Foram comentadas as atividades de Palestras e Minicursos.
Foi questionado sobre a adesão dos alunos com os minicursos, Lucas respondeu
que no último ano decaiu, e João Paulo respondeu que não sentiu decadência,
comentando que fora o minicurso de Gitlab todos estavam bem lotados. Victor
comenta que é necessário que os professores criem uma visão de que a SECCOM não
é apenas para computação. Júlia comenta sobre palestras que deram bastante
público, citando a de Biocomputação e a da RD\@. Lucas levantou sobre a Feira
de Laboratórios e Thales complementou que não deu muito certo, já que foram
apenas 3 laboratórios e poucos alunos. Lucas também levantou sobre a Feira
Empresarial e as competições (Maratona e CTF), que deram bastante certo. Victor
comenta que seria interessante ter Lightning Talks, possivelmente entre
palestras. João Paulo comenta que tinha a ideia de trocar a Feira de
Laboratórios por Lightning Talks para laboratórios, tendo 8 a 10 minutos para
cada. Luis responde que, se metade dos laboratórios aceitarem, seriam já cerca
de 100 minutos. Vanessa fala sobre fazer a divulgação (chamadas de palestrantes
e etc.) bem mais cedo do que antes, pois no ano passado foi chamada muito em
cima da hora.

Luis resume que os espaços necessários são:

\begin{itemize}
    \item Espaço para minicursos;
    \item Espaço para palestras;
    \item Espaço para a feira empresarial.
\end{itemize}

Sobre a feira empresarial, Lucas comenta que as empresas ficaram impressionadas
com o movimento (chegando a comparar com a Workweek), mas também ressalta que
houve uma queda no movimento. Júlia argumenta que pode ter sido tempo demais de
feira (em dias), afinal, segundo Lucas, o movimento diminuiu mais nos últimos
dias chegando ao ponto de empresas não irem no último dia.

Lucas fala sobre a ordem de prioridades para reserva do Teixeirão, que é
interessante um professor do EEL para fazer a reserva. Sobre a Reitoria, Thales
comentou que é um espaço bom, porém muito grande e afastado do CTC\@. Vanessa
comentou sobre a vantagem de que as empresas não necessariamente contratam
pessoas apenas da área tecnológica, e que às vezes tem vagas de outras área. É
levantado que é difícil tirar conclusões da SECCOM 2017, ocorrida na Reitoria,
pois foi na mesma semana em que o reitor na época faleceu (potencialmente
impactando o movimento). É concordado entre os membros da reunião de que o EPS
é o mais viável, com a ressalva feita pelo Thales de que é complicado de
controlar o coffee-break. Luis pergunta se o Teixeirão é uma boa ideia e Thales
responde que foi bom, e que o coffee-break foi feito em uma sala ao lado. Júlia
comenta que um problema do Teixeirão é que teremos que falar com a Ampera, que
entrará em contato com um professor responsável pela reserva, e aí sim este
entrará em contato com a CO\@. Luis se mostrou contra fazer a reserva na
Reitoria. É comentado também que o Teixeirão tem a vantagem de que o
coffee-break e a Feira Empresarial ficam próximos.

\textbf{A prioridade definida então é entrar em contato com alguém sobre a
reserva do Teixeirão.} Júlia pergunta se é bom já tentar reservar o EPS como
segunda prioridade, e Luis responde que é importante primeiro ver sobre a lista
de prioridades. Lucas pergunta então se fica definido se a Feira Empresarial
ficará no hall do CTC, e Luis questiona sobre a poluição visual, especialmente
durante a troca de aulas dos alunos (para ir ao banheiro ou trocar de sala, por
exemplo). Também é comentado sobre o tempo da feira, se seria feita em menos
dias. Júlia comenta que poderiam ser 2 dias (em vez de 4). Foi sugerido também
separar a feira em períodos (algumas empresas em um período, e outras no
outro). Luis comenta que é bom reduzir o espaço utilizado no hall.

\pauta{Organização interna}

Thales pergunta se a CO será organizada em frontes. Luis comenta que não vê
necessidade, e Thales responde que vê necessidade por conta de ter menos
pessoas na reunião. É definido que esse assunto será melhor discutido em uma
próxima reunião.

\pauta{Escolha das datas}

João Paulo fala que a metade de outubro (na SECCOM 2018) foi tranquilo. Lucas
fala que sente que a SECCOM o fez não estudar para a graduação por muito tempo.
Foi levantado sobre a 1ª semana de Outubro, e que é necessário ver sobre datas
de provas. Também foi comentado que se a SECCOM for inserida no calendário
acadêmico, os professores não deveriam marcar provas nela. Thales questiona
sobre feriados em Outubro e Vanessa responde que tem apenas dia 12 (Dia das
Crianças). Lucas pergunta se há algum feriado antes e Vanessa responde que tem
também dia 7 de Setembro (e portanto não interfere). \textbf{Fica sugerido
então fazer a SECCOM na primeira semana de Outubro (30 de Setembro a 4 de
Outubro).}

\pauta{Planos de patrocínio}

Lucas fala que irá mandar para todos o PDF com planos de patrocínios de anos
anteriores. Júlia pergunta se há uma lista de empresa de empresas que já
patrocinaram, e Lucas responde que o que há é uma lista de empresas com as
quais foi entrado em contato, só ressalta que a lista contém 80 nomes dos 9 que
patrocinaram. \textbf{Lucas sugere que olhem a lista e vejam se há alguma outra
empresa interessante.} Thales comentou que a aproximação com as empresas que
teve nas edições em que participou foi dando ideia e mais detalhes de como ia
ser feita a SECCOM\@. Vanessa comenta que primeiro temos que ter garantia do
professor responsável, depois de termos o espaço físico definido, e então ver
que atividades teremos.

\pauta{Próximas pautas}

Foram elencadas as seguintes pautas para as próximas reuniões:

\begin{itemize}
    \item Organização Interna;
    \item Cronograma geral;
    \item Marketing da SECCOM\@;
    \item Planejamento do planejamento estratégico.
\end{itemize}

\presentes{%
    João Paulo T\@.,
    Júlia Nakayama,
    Lucas Sousa,
    Luis Oswaldo Ganoza,
    Marcelo Brosowicz,
    Thales Alexandre Zirbel Hübner,
    Vanessa Cunha,
    Victor Goulart.
}

\end{document}
